
\section*{Dissertatsiyaning umumiy tavsifi}

\subsection{Mavzuning dolzarbligi.}
O'zbekiston Respublikasi Prezidentining 
21.03.2012y. PQ-1730 �Zamonaviy axborot - kommunikatsiya texnologiyalarini 
yanada joriy etish va rivojlantirish chora-tadbirlari to'g'risida�gi qarori va 
ushbu qarorni amaldagi tadbig'ini ta'minlash maqsadida 2012-2014 yillarga 
mo'ljallangan dasturni sifatli bajarilishiga zamin yaratish bugungi kunning 
dolzarb masalalaridan biri hisoblanadi. Chunki axborot-kommunikatsiya 
texnologiyalar (AKT)ni jamiyatning turli sohalarida keng joriy 
etilishining izchil sur'atlarda olib borilishi munosabati bilan tub 
islohatlar o'tkazish va mazkur sohadagi faoliyatni tartibga soluvchi qonun 
hujjatlarini takomillashtirish zaruriyati yuzaga keldi.

Hozirda fan taraqqiyoti tadqiq qilinayotgan obyektlar yoki obyektlar
jamlanmasi tarkibi murakkabligining ortib borishi kuzatilmoqda. 
Ob'ektlar haqidagi axborotning oshib borishi, ushbu axborotni saqlash 
va qayta ishlash uchun kompyuter vositalarining qo'llanilishi, amaliyotda 
to'planagan axborotlar tarkibiga yashiringan qonuniyatlarni aniqlash 
hamda ijobiy hal qilish masalalari yanada dolzarb bo'lib bormoqda. 
Ma'lumotlarni saqlash uchun ma'lumotlar bazasini (MB) yaratish ehtiyoji 
tug'ilib, uning vositasida to'plangan ma'lumotlarga samarali ishlov berish 
yondashuvlari va tizimlarini ishlab chiqish zarur. Bunday mexanizmli 
vositalarni ishlab chiqishda so'ngi vaqtlarda shiddat bilan rivojlanayotgan 
ma'lumotlarni intellektual tahlili usullaridan unumli foydalanish 
maqsadga muvofiq. Ma'lumki, ma'lumotlarni intellektual tahlilining 
asosiy masalalari kassifikatsiya, regressiya, bashoratlash, assotsiativ 
qoidalarni qidirish, klasterizatsiya masalalarini qamrab oladi. 

Biologik tasniflar va ma'lumotlar bazasi yuzlab, minglab va undan 
ham ko'p turli xil darajadagi yuzlab belgilarga ega bo'lgan taksonlardan 
tashkil topadi. Kompyuter aniqlagichlari uchun bunday katta hajmdagi MB 
asosida tasniflashni amalga oshirish juda muhim hisoblanib, bunda qaror 
qabul qilish uchun eng maqbul muqobilni olish talab etiladi.
 
O'simliklar, xayvonlar va zamburug'lar aniqlagichlari 300 
yildan beri biolog olimlar 
tomonidan tadqiq qilib kelmoqda va ularning tuzilish usullari muhokama 
qilinmoqda, ularning biologik tasniflarini sistematik kalitlar 
yordamida dastlabki avtomatizatsiyalash o'tgan asrlardagi hisoblash 
texnikasi yordamida amalga oshirilgan. Tasniflash juda ko'pgina biologik 
tadqiqotlar jarayonida soha olimlarining amaliyotida muhim ahamiyat kasb 
etadi.

O'zbekiston biolog olimlari uchun ham yuqoridagi kabi muammolar vujudga 
kelmoqda. Shuning uchun, tahlillardan kelib 
chiqib qisman pretsendentlikka asoslangan timsollarni 
aniqlash algoritmi asosida biologik tizimlarni tasniflash va tadqiq qilish 
tizimini tashkil etish tadqiqotning dolzarbligini belgilaydi.

\textbf{Muammoning o'rganilganlik darajasi}
Filogenetik ma'lumotlarni o'rganish, tadqiq qilish va ularning
algoritmlarini ishlab chiqish, shuningdek, shu soha vakillari
uchun dastruiy ta'minoti ishlab chiqish masalalari 
bo'yicha bir qator xorijiy mamlakatlar olimlari va 
respublikamiz yetakchi olimlari ilmiy-nazariy va ilmiy-amaliy tadqiqot 
ishlarini olib borishgan. Fan va texnika borasida erishilgan natijalar: 
taklif va tavsiyalar, ishlanmalar, dasturiy vositalar, tuzilmalar va 
tizimlar jamiyatning turli sohalarida samarali qo'llanilib kelinmoqda.

Dunyo miqyosida filogenetik ma'lumotlarni tahlil qilish tizimini yaratish va 
rivojlantirish bo'yicha bir qator dasturiy mahsulotlar 
ishlab chiqilgan: PHYLIP, PAUP, MacClade, TREECON, Spectrum, TREEMAP,
DNA Stacks, GeneDoc, Seq-Gen, Phylodendron, GeneTree, PASSML, BioEdit,
MEGA, PAST, ClustalW va h.k.

O'rganilgan nazariy manbalar va qayd etilgan tadqiqot ishlari 
tahlilidan ma'lum bo'ldiki, ular filogenetik ma'lumotlar tahlil
qilishda ma'lum bir algoritmlar asosida ishlaydi va bularning 
hech birida baholash algoritmlaridan foydalanilmagan. Qolaversa, 
respublikamiz olimlari bu dasturiy ta'minotlardan foydalanishda
bir qancha muammolarga duch kelishmoqda.
%TODO - muammo

\textbf{Tadqiqot maqsadi.}
Filogenetik ma'lumotlarni tahlil qilish prosedurasini tashkil qilish.
Tahlil davomida vaqt va kompyuter resurslaridan samarali foydalanishni,
natijalar to'liq va ishonchli bo'lishini ta'minlash.

\textbf{Tadqiqot vazifalari.}
Qo'yilgan maqsadni amalga oshirish uchun 
quyidagi vazifalarni bajarish talab etiladi: 

\begin{itemize}
	\item Filogenetik ma'lumotlarni o'rganish, tadqiq qilish 
	uchun ishlab chiqilgan algoritmlar va dastruiy 
	ta'minotlarni tahlil qilish;
	\item Filogenetik ma'lumotlarni tahlil qilish tizimining 
	o'ziga xos hususiyatlari va takomillashtirish omillarini 
	aniqlash;
	\item Filogenetik ma'lumotlarni tahlil qilishda baholash 
	algoritmlaridan foydalanish tamoyillarini belgilash;
	\item Filogenetik ma'lumotlarni tahlil qilish prosedurasini
	ishlab chiqish;
	\item Filogenetik ma'lumotlarni tahlil qilish dasturiy
	ta'minotini ishlab chiqish;	 
\end{itemize}

\textbf{Tadqiqot obyekti.}\\
Filogenetik ma'lumotlar, ularni tahlil qilish usullari,
modellari, algoritmlari va dasturiy ta'minotlari tadqiqot
obyektlari hisoblanadi.

\textbf{Tadqiqot predmeti.}
Filogenetik ma'lumotlar tahlil qilishning modellari
va dasturiy vositalari.

\textbf{Tadqiqot usullari.} %TODO - ?
Tizimli tahlil, boshqarish nazariyasi va hisoblash 
usullari, dasturlash texnologiyasi va usullari, modellashtirish usullari, 
timsollarni aniqlash usullari.

\textbf{Tadqiqot gipotezasi.} %TODO - is it correct
Filogenetik ma'lumotlar tahlil qilishda baholash 
algoritmlarini qo'llash proseduraning tezkorligini va aniqlik
darajasini oshiradi.

\textbf{Himoyaga olib chiqilayotgan asosiy holatlar:}

- filogenetik ma'lumotlar tahlil qilish 
mexanizmlarining tahlili;

- filogenetik ma'lumotlarni tahlil qilishning 
o'ziga hos xusussiyatlaridan kelib chiqqan holda
mukammal axborot texnologiyalari va dasturiy 
vositalar yordamida takomillashtirish omillari;

- filogenetik ma'lumotlarni tahlil qilish tizimini\\ 
shakklantirishning mavjud dasturiy ta'minoti taqqosiy tahlili;

- katta hajmdagi ma'lumotlarni tahlil qilishda
qimmatli vaqt va moddiy sarf xarajatlarining tejash mexanizmi;


\textbf{Ilmiy yangiligi.}
Tadqiqot ishida asosiy ilmiy yangiliklar quyidagilardan iborat:

- filogenetik ma'lumotlarni tahlil qilishda
baholash algoritmlaridan foydalanildi;

- filogenetik ma'lumotlarni tahlil qilish prosedurasi modeli
va shu model asosida dasturiy vosita ishlab chiqildi;

- filogenetik ma'lumotlar saqlashga mo'njallangan 
ma'lumotlar bazasi ishlab chiqildi;

\textbf{Tadqiqot natijalarining ilmiy va amaliy ahamiyati.}

Filogenetik ma'lumotlarni tahlil qilishda baholash algoritmlarini\\
qo'llashning amaliy ahamiyati aniqlandi.

Filogenetik ma'lumotlarni tahlil qilishnig mavjud algoritmlar natijalari va 
baholash algoritmlari natijalari taqqoslandi.

Filogenetik daraxtlar qurish, proteinlar ketma-ketligini moslashtirish,
filogenetik ma'lumotlarni tahlil qilish mexanizmlari ishlab chiqildi.

\textbf{Natijalarining joriy qilinishi.}

Ishlab chiqilgan amaliy dastur O'zbekiston Respublikasi\\
fanlar akademiyasi qoshidagi ``Botanika'' ilmiy-ishlab chiqarish
markazi hodimlari tomonidan ijobiy baholandi va %TODO - tashkilot qushish, Yusufjon aka 
foydalanishga topshirildi.

\textbf{Natijalarning e'lon qilinganligi.}
Dissertatsiya ishining natijalari asosida 2 ta ilmiy 
ishlar nashr qilingan. Ular 1 ilmiy jurnal maqolasi,
1 ilmiy anjuman tezislari hamda EHM uchun yaratilgan 
dasturiy vositalardir. 

\textbf{Dissertatsiyaning tuzilishi va hajmi.}
Dissertatsiya ishi kirish, 3 ta bob, xulosa, foydalanilgan adabiyotlar 
ro'yxati hamda ilovalardan tashkil topgan. Ishning asosiy matni 40 getda %TODO betni tugirla
o'z ifodasini topgan. 

\vspace{5mm}

\begin{center}
\subsection*{\large DISSERTATSIYANING ASOSIY MAZMUNI}
\end{center}

\textbf{Kirishda} dissertatsiya ishining dolzarbligi asoslangan, 
tadqiqot mazmun mohiyati to'la yoritilgan. Muammoning 
o'rganilganlik darajasi, dissertatsiya ishining ilmiy-tadqiqot 
ishlari bilan bog'liqligi, tadqiqot maqsadi va vazifalari, 
tadqiqot ob�ekti va predmeti, tadqiqot usullari aniqlangan. 
Shu bilan birga himoyaga olib chiqiladigan asosiy holatlar, 
ishning ilmiy yangiligi, tadqiqot natijalarining ilmiy va amaliy ahamiyati, 
tadqiqot natijalarining joriy qilinishi, ishning sinovdan o'tishi, 
natijalarning nashr qilinganligi ham keltirilgan. Bundan tashqari 
tizimni joriy qilish samaradorligi hamda dissertatsiyaning 
tuzilishi va hajmi bayon etilgan. 

\textbf{Birinchi bobda} keltiriladigan ma'lumotlar,
filogenetika fanining asosiy vazifasi 
filogenetik daraxt qurish va filogenetik 
tahlilining natijasi daraxt ko'rinishidagi ma'lumotlar 
klassifikatsiyasi haqida gap boradi.

Filogenetikaga tegishli asosiy
tushuncha va atamalar keltirilgan. Xusussan, graflar, daraxtlar, 
filogenetikaning umumiy tavsifi va o'ziga hos xususiyatlari, daraxt shaklidagi 
graflar va ularning turlari va filogenetik daraxt ifodalanish turlari haqida 
ma'lumotlar keltirilgan.

Toshkent axborot texnologiyalari universiteti xuzuridagi 
Dasturiy maxsulotlar va apparat-dasturiy majmualar yaratish markazida
ishlab chiqilgan va respublikamizda ko'plab sohalarda o'z aksini
topgan baholash algoritmlari (BA) haqida ma'lumotlar keltirilgan.
Baholash algoritmlarining asosiy xossalari va tushunchalari, 
afvzalliklari va filogenetik ma'lumotlarni tahlil qilish tizimini 
yaratishdagi samarasi haqida gap boradi. 

Mavjud filogenetik daraxt qurish mexanizmlari va 
filogenetik daraxt qurish algoritmllari haqida ma'lumotlar kelgan.

Keltirib o'tilgan ma'lumot va muammolardan kelib chiqib disertatsiya 
asosiy ishning maqsadi aniqlangan. Maqsadni amalga oshirish uchun
uni asosiy qismlarga ajratilgan.

\textbf{Ikkinchi bobda}
Belgilangan qadamlar

\textbf{Uchinchi bobda}

\newpage

\begin{center}
\subsection*{\large XULOSA}
\end{center}

\newpage

\begin{center}
\subsection*{\large E'LON QILINGAN ISHLAR RO'YXATI}
\end{center}
